\documentclass[a4paper,12pt]{article}

\usepackage[latin1]{inputenc}
\usepackage[english]{babel}
\usepackage[dvips]{graphicx,color}
\usepackage{calc}

\newcommand{\fig}[3]
{
	\begin{figure}[ht!]
		\centering
		\includegraphics[scale=0.4]{#1}
		\caption{#2.\label{#3}}
	\end{figure}
}

\newcommand{\PICTURE}[4]
{
	\begin{figure}[ht!]\centering\begin{picture}(#1)#2\end{picture}
	\caption{#3.\label{#4}}\end{figure}
}

\newcommand{\TABLE}[4]
{
	\begin{table}[ht!]\centering
	\begin{tabular}{#1}\hline#2\\\hline\end{tabular}
	\caption{#3.\label{#4}}\end{table}
}

\newcommand{\swigs}{\emph{SWIGS}}
\newcommand{\IT}[1]{{\sl ``#1''}}

\title
{
	{\bf\Large {\swigs} reference manual}\\
	{\large	English version 1.3.0 for Windows}
}
\author
{
	{\bf Author:} Javier Burguete Tolosa\\
	{\small Copyright~\copyright~2005,~2014 Javier Burguete.
	All right reserved}
}
\date{\today}

\begin{document}

\maketitle

\tableofcontents
\clearpage

\setlength{\parskip}{\baselineskip / 2}

\section{Introduction}

The programme {\swigs} is an interactive and easy-to-use tool specially designed
to perform computer simulation of one-dimensional transient flows in all kind of
river and channel beds. 

The code has been written in C language and based on the graphic facilities
offered by \emph{GTK+} and \emph{FreeGLUT} libraries, on the
internationalization offered by \emph{Gettext} and on the \emph{libxml2} library
for the XML files management. These are all free software libraries under LGPL
license. In the Windows version the code has been compiled with \emph{MinGW}.

Next sections describe the Windows version. Windows is a trademark of Microsoft
Corporation.

\section{Internationalization}

In order to avoid conflicts with formats in the input and output data files when
using different regional configurations, the real numbers format has been forced
to the international standard. Therefore, the character \IT{.} has to be used to
indicate the decimal point. The character \IT{e}, indicates the exponential of
the number. On the other hand, all units have to be specified in the
International Unit System. 

\section{Building from source code}

To build from source code follows the next steps:
\begin{itemize}
	\item Download the latest JB library
		(\emph{https://github.com/jburguete/jb})
	\item cd VERSION\_NUMBER (i.e. cd 1.3.0)
	\item link the latest JB library to jb (i.e. ln -s
		PATH\_TO\_THE\_JB\_LIBRARY/1.8.3 jb)
	\item aclocal
	\item autoconf
	\item automake --add-missing
	\item ./configure
	\item make
	\item strip swigs swigsbin (optional: to make a final version)
\end{itemize}

On Microsoft Windows operative systems, previously you have to install
MSYS/MinGW (\emph{http://www.mingw.org}) and the required libraries and
utilities. You can follow detailed instructions in
\emph{https://github.com/jburguete/MinGW-64-Make}

\section{Main window}

In order to start running the program {\swigs} the file \IT{guad1d.exe} must
be executed. The screen shows then a window like the one illustrated in
figure \ref{FigSimulador}.
\fig{Simulator.ps}{Main window}{FigSimulador}

In order to finish the simulation the button \IT{Exit} must be pressed or the
menu option \IT{$\underline{C}$hannel$\rightarrow E\underline{x}$it} can be used.

To get in contact with the program developper the button \IT{Help} or the menu
option \IT{$\underline{H}$elp$\rightarrow\underline{A}$bout} can be used, showing
a dialog box as the one in figure \ref{FigAyuda}.
\fig{Help.ps}{Dialog box}{FigAyuda}

\section{The channel}

To start using the program, the button \IT{Open channel} from the main window or
the menu option \IT{$\underline{C}$hannel$\rightarrow\underline{O}$pen} mist be
pressed, leading to a dialog box like the one in figure \ref{FigCauce} which
allows to open the bed file. 
\fig{Channel.ps}{Dialog box to open the bed}{FigCauce}

The option \IT{O$\underline{p}$en channel} in this dialog box must be used to
choose a bed definition file. The format of this file will be described in next
sections.

The option \IT{$\underline{S}$pecify roughness} also appears in the dialog box.
When activated, the program ignores the locally defined roughness coefficients
and applies a global value to the roughness coefficient.

\subsection{Channel file format}

This file consists of a series of modular structures nested in a tree form as
displayed on figure \ref{FigFCauce}.
\PICTURE{300,300}
{
	\put(-20,80){Channel}
	\put(35,82){\line(1,6){20}}
	\put(35,82){\line(1,1){20}}
	\put(35,82){\line(2,-1){20}}
	\put(35,82){\line(2,-5){20}}
	\put(60,200){Geometry}
	\put(125,202){\line(2,5){20}}
	\put(125,202){\line(2,-5){20}}
	\put(150,250){Cross section$_1$}
	\put(225,252){\line(1,2){10}}
	\put(225,252){\line(1,-2){10}}
	\put(240,270){Transient$_{1_1}$}
	\put(270,250){$\vdots$}
	\put(240,230){Transient$_{m_1}$}
	\put(170,200){$\vdots$}
	\put(150,150){Cross section$_n$}
	\put(225,152){\line(1,2){10}}
	\put(225,152){\line(1,-2){10}}
	\put(240,170){Transient$_{1_n}$}
	\put(270,150){$\vdots$}
	\put(240,130){Transient$_{m_n}$}
	\put(60,100){Initial conditions}
	\put(60,70){Boundary condition$_1$}
	\put(115,50){$\vdots$}
	\put(60,30){Boundary condition$_k$}
}{Structure of the channel file}{FigFCauce}

The channel file format follows a standard XML label system as displayed in table \ref{TabCauce}.
\TABLE{l}
{
$<$?xml version="1.0"?$>$\\
$<$channel$>$\\
\hspace{2cm}$<$channel\_geometry$>$\\
	\hspace{4cm}$<$cross\_section ...$>$\\
		\hspace{6cm}$<$transient\_section ...$>$\\
			\hspace{7cm}$\vdots$\\
		\hspace{6cm}$<$/transient\_section$>$\\
		\hspace{6cm}$\vdots$\\
	\hspace{4cm}$<$/cross\_section$>$\\
	\hspace{4cm}$\vdots$\\
\hspace{2cm}$<$/channel\_geometry$>$\\
\hspace{2cm}$<$initial\_flow ...$>$\\
	\hspace{3cm}$\vdots$\\
\hspace{2cm}$<$/initial\_flow$>$\\
\hspace{2cm}$<$inlet\_flow ...$>$\\
	\hspace{3cm}$\vdots$\\
\hspace{2cm}$<$/inlet\_flow$>$\\
\hspace{2cm}$<$inner\_flow ...$>$\\
	\hspace{3cm}$\vdots$\\
\hspace{2cm}$<$/inner\_flow$>$\\
\hspace{2cm}$\vdots$\\
\hspace{2cm}$<$outlet\_flow ...$>$\\
	\hspace{3cm}$\vdots$\\
\hspace{2cm}$<$/outlet\_flow$>$\\
$<$/channel$>$
}{Format of the channel file}{TabCauce}

Where the following is defined:
\begin{description}
\item\emph{channel}: the bed form.
\item\emph{channel\_geometry}: the bed form geometry.
\item\emph{cross\_section}: a bed cross section.
\item\emph{transient\_section}: a transient bed form at a given time.
\item\emph{initial\_flow}: the initial flow conditions.
\item\emph{inlet\_flow}: the inflow discharge.
\item\emph{inner\_flow}: an interior boundary condition.
\item\emph{outlet\_flow}: the outflow discharge.
\end{description}

The required format to specify the above structures of the bed form file is
described in next subsections \ref{SubsecTransversal}-\ref{SubsecBoundary}.

\subsection{Structure of a transient section\label{SubsecTransversal}}

In {\swigs} the transient sections define the bed geometry of a channel cross
section at a given time. They have a general structure as in table
\ref{TabTransversal},
\TABLE{ccccc}
{
	\multicolumn{5}{l}{$<$transient\_section name=``label'' propierties...$>$}\\
	&$y_1$&$z_1$&$\mu_1$&$t_1$\\&$\vdots$&$\vdots$&$\vdots$&$\vdots$\\
	&$y_i$&$z_i$&$\mu_i$&$t_i$\\&$\vdots$&$\vdots$&$\vdots$&$\vdots$\\
	&$y_{n-1}$&$z_{n-1}$&$\mu_{n-1}$&$t_{n-1}$\\&$y_n$&$z_n$\\
	\multicolumn{5}{l}{$<$/transient\_section$>$}
}{General format of the structure of a transient section}{TabTransversal}

Where:
\begin{description}
\item\emph{label}: is the name of the transient section. It is used as identifying label and all of them must have a different name. 
\item\emph{properties}: three properties can be defined at every transient section.
\begin{description}
\item\emph{time=``t''}: where \emph{t} is the time assigned to the transient cross section. By default the value zero is assumed.  
\item\emph{velocity=``u''}: where \emph{u} is the bottom velocity, allowing moving beds. By default the value zero is assumed. 
\item\emph{contraction=``$\beta$''}: where $\beta$ is the contraction-expansion coefficient. By default the value zero is assumed.
\end{description}
\item $y_i$: Transversal coordinate of the points defining the transversal. 
\item $z_i$: Vertical coordinate of the points defining the transversal. 
\item $\mu_i$: Roughness coefficient applied to the length between
points $i$ and $i+1$.
\item $t_i$: Roughness model used (Manning, Ch\`ezy,
Darcy, ...) in the length between points $i$ and $i+1$.
\end{description}
In the present version only Manning's law is available ($t_i=0$).

For a bed cross section such as the one represented in
figure \ref{FigTransversal},
\PICTURE{280,150}
{
	\put(265,10){$y$}
	\put(20,20){\vector(1,0){240}}
	\put(10,135){$z$}
	\put(20,20){\vector(0,1){110}}
	\put(20,100){$(40,90)$}
	\put(40,90){\line(4,-1){80}}
	\put(80,80){\vector(1,4){7}}
	\put(65,110){$\mu=$0.03}
	\put(40,90){\circle*{3}}
	\put(100,80){$(80,80)$}
	\put(120,70){\line(1,-1){20}}
	\put(130,60){\vector(-1,0){30}}
	\put(55,57){$\mu=$0.02}
	\put(120,70){\circle*{3}}
	\put(120,40){$(90,70)$}
	\put(140,50){\line(2,1){40}}
	\put(160,60){\vector(1,0){30}}
	\put(195,57){$\mu=$0.04}
	\put(140,50){\circle*{3}}
	\put(160,80){$(110,80)$}
	\put(180,70){\line(3,1){60}}
	\put(210,80){\vector(-1,3){10}}
	\put(180,112){$\mu=$0.03}
	\put(180,70){\circle*{3}}
	\put(240,90){\circle*{3}}
	\put(220,100){$(140,90)$}
}
{Standard transversal where $\mu$ is the Manning's number in the segment}
{FigTransversal}
the structure of the transversal must be that of table \ref{TabTransversalC}.
\TABLE{ccccc}
{
	\multicolumn{5}{l}{$<$transient\_section name="label"$>$}\\
	&40&90&0.03&0\\&80&80&0.02&0\\&90&70&0.04&0\\&110&80&0.03&0\\&140&90\\
	\multicolumn{5}{l}{$<$/transient\_section$>$}
}
{Format of the transversal structure represented in figure \ref{FigTransversal}}
{TabTransversalC}

In the case of having to represent transversals formed by closed regions such as
bridges, the points must be defined following the ordering of a graph including
the closed regions. This is illustrated in the example presented in
figure \ref{FigTCerrado}.
\PICTURE{280,200}
{
	\put(265,10){$y$}
	\put(20,20){\vector(1,0){240}}
	\put(10,185){$z$}
	\put(20,20){\vector(0,1){160}}
	\put(40,160){\line(3,-1){30}}
	\put(40,160){\circle*{3}}
	\put(37,164){$1$}
	\put(70,150){\line(1,-1){10}}
	\put(70,150){\circle*{3}}
	\put(67,154){$2$}
	\put(80,140){\line(1,0){40}}
	\put(80,140){\circle*{3}}
	\put(77,144){$3$}
	\put(120,140){\line(0,-1){50}}
	\put(120,140){\circle*{3}}
	\put(112,144){$4$}
	\put(122,144){$10$}
	\put(120,90){\line(-4,-1){40}}
	\put(120,90){\circle*{3}}
	\put(112,94){$5$}
	\put(122,94){$9$}
	\put(80,80){\line(1,-2){20}}
	\put(80,80){\circle*{3}}
	\put(77,84){$6$}
	\put(100,40){\line(3,1){60}}
	\put(100,40){\circle*{3}}
	\put(97,44){$7$}
	\put(160,60){\line(-4,3){40}}
	\put(160,60){\circle*{3}}
	\put(157,64){$8$}
	\put(120,140){\line(4,-1){80}}
	\put(200,120){\line(1,2){20}}
	\put(200,120){\circle*{3}}
	\put(194,124){$11$}
	\put(220,160){\circle*{3}}
	\put(214,164){$12$}

}
{Ordering of the points used to define closed transversals.}{FigTCerrado}

\newpage
\subsection{Structure of a cross section}

In {\swigs} the cross sections define the possible variations in time of the form
of the bed cross section as well as their location in the plane. They have a
general structure as shown in table \ref{TabSeccion}.
\TABLE{cl}
{
	\multicolumn{2}{l}{$<$cross\_section name=``label'' x=``$x_0$'' y=``$y_0$''
		angle=``$\alpha$''$>$}\\
	&Transient$_1$\\
	&$\vdots$\\
	&Transient$_n$\\
	\multicolumn{2}{l}{$<$/cross\_section$>$}
}{General format of the structure of a cross section}{TabSeccion}

Where:
\begin{description}
\item\emph{label}: Name of the cross section. No gaps are allowed. It is used as
an identification label.
\item$x_0$: $x$ coordinate of the location of the transient cross
sections on the plane.
\item$y_0$: $y$ coordinate of the location of the transient cross
sections on the plane. By default the value zero is assumed.
\item$\alpha$: Angle (in degrees) between the transversal direction
and the $x$ axis direction. By default the value $90\,^{\circ}$ is assumed.
\item\emph{Transient$_i$}: is the structure of the transient section defining the geometry of the bed cross section from time $t_i$ assigned to that structure.
\end{description}

As an example, for a bed cross section undergoing two changes in time as
represented on figure \ref{FigSeccion},
\PICTURE{280,150}
{
	\put(265,10){$y$}
	\put(20,20){\vector(1,0){240}}
	\put(10,135){$z$}
	\put(20,20){\vector(0,1){110}}
	\put(40,90){\line(4,-1){80}}
	\put(120,70){\line(1,-1){20}}
	\put(140,50){\line(2,1){40}}
	\put(180,70){\line(3,1){60}}
	\put(160,40){Transient$_3$ $(t_3=40)$}
	\put(210,50){\vector(0,1){30}}
	\put(60,100){\line(3,-1){60}}
	\put(120,80){\line(4,-5){20}}
	\put(140,55){\line(1,1){30}}
	\put(170,85){\line(6,1){30}}
	\put(160,105){Transient$_2$ $(t_2=20)$}
	\put(210,100){\vector(-1,-1){10}}
	\put(80,110){\line(2,-1){50}}
	\put(130,85){\line(1,-1){15}}
	\put(145,70){\line(1,1){30}}
	\put(40,115){Transient$_1$ $(t_1=0)$}
	\put(90,110){\vector(1,-1){10}}
}
{Geometry of a bed cross section in which two variations in time take
place}{FigSeccion}
and located at a position on the plane with coordinates $x=20$, $y=40$ with an
angle $\alpha=45\,^{\circ}$, the structure of the section would be as shown in
table \ref{TabSeccionC},
\TABLE{cl}
{
	\multicolumn{2}{l}
	{$<$cross\_section name=``label'' x=``20'' y=``40'' angle=``45''$>$}\\
	&$<$transient\_section name=``transient$_1$'' ...$>$\\
	&$\cdots$\\
	&$<$/transient\_section$>$\\
	&$<$transient\_section name=``transient$_2$'' time=``20''...$>$\\
	&$\cdots$\\
	&$<$/transient\_section$>$\\
	&$<$transient\_section name=``transient$_3$'' time=``40''...$>$\\
	&$\cdots$\\
	&$<$/transient\_section$>$\\
	\multicolumn{2}{l}{$<$/cross\_section$>$}
}
{Format of the structure of the cross section represented in
figure \ref{FigSeccion}}{TabSeccionC}
where \emph{transversal$_i$} represents the data structure defining transversal
$i$.

It is worth noting that the program performs the geometry changes in the
transversal cross section in a discontinuous way. In this example the gross
sectional geometry will be defined in time as follows:
\begin{itemize}
\item $t\in(-\infty,20)\Rightarrow$ Transient$_1$.
\item $t\in[20,40)\Rightarrow$ Transient$_2$.
\item $t\in[40,\infty)\Rightarrow$ Transient$_3$.
\end{itemize}

Let us assume for instance a cross section in which the transversals will be
located at the plane position $x_0$, $y_0$ and forming an angle $\alpha$ with
the horizontal. The way to place in the plane the points of a transient cross
section of transversal coordinates $y_i$ is sketched in figure
\ref{FigSeccionPlano}.
\PICTURE{280,180}
{
	\put(265,10){$x$}
	\put(20,20){\vector(1,0){240}}
	\put(10,165){$y$}
	\put(20,20){\vector(0,1){140}}
	\put(80,60){\circle*{3}}
	\put(60,50){$(x_0,y_0)$}
	\put(80,60){\line(2,1){100}}
	\put(80,60){\line(1,0){40}}
	\qbezier(100,60)(98,67)(96,68)
	\put(105,65){$\alpha$}
	\put(120,80){\circle*{3}}
	\put(140,90){\circle*{3}}
	\put(180,110){\circle*{3}}
	\qbezier[5](80,60)(75,70)(70,80)
	\qbezier[5](120,80)(115,90)(110,100)
	\qbezier[5](140,90)(135,100)(130,110)
	\qbezier[5](180,110)(175,120)(170,130)
	\put(70,80){\vector(2,1){40}}
	\put(70,80){\vector(2,1){60}}
	\put(70,80){\vector(2,1){100}}
	\put(95,100){$y_1$}
	\put(115,110){$y_2$}
	\put(155,130){$y_3$}
}
{Scketch of the way to place on the plane the points of a transversal defined at
a section}{FigSeccionPlano}

The reference point $(x_0,y_0)$ of transversal location can be placed at any
transversal segment. However, the choice of the point location is not arbitrary
since the program internally calculates the distances among cross sections
sections as the distance between those reference points. It is recommended to
locate those points at the lowest transversal point.

\subsection{Structure of the channel geometry}

In the program {\swigs} a channel geometry is the set of connected cross sections.
The general structure is as displayed in table \ref{TabGeometria}.
\TABLE{cc}
{
	\multicolumn{2}{l}{$<$channel\_geometry$>$}\\
	&Section$_1$\\&\vdots\\&Section$_n$\\
	\multicolumn{2}{l}{$<$/channel\_geometry$>$}
}{General structure of the channel geometry}{TabGeometria}

Where
\begin{description}
\item\emph{Section$_i$}: Structure of cross section $i$.
\end{description}

Even though for calculation purposes the program is fully symmetric, for
representation purposes section 1 is considered the upstream section and section
$n$ the downstream section. The $x$ coordinate used for the longitudinal
representation is the distance to the nest upstream cross section following the
segments that join the reference points $(x_0,y_0)$ of every cross section.

\subsection{Structure of the initial conditions}

The bed geometry is associated to the corresponding initial and boundary conditions. The general structure of the initial conditions is
specified as displayed in table \ref{TabFCI},
\TABLE{ccc}
{
	\multicolumn{2}{l}{$<$initial\_flow type=``type''$>$}\\
	&Parameters\\
	\multicolumn{2}{l}{$<$/initial\_flow$>$}
}
{Structure of the initial conditions}{TabFCI}

where:
\begin{description}
\item\emph{Type}: Code of the type of initial condition. The codes included
in the present version are:
\begin{description}
\item\emph{dry}: Dry bed.
\item\emph{steady}: Steady flow conditions.
\item\emph{xqh}: List of values $(x_i,Q_i,h_i)$ where:
\begin{description}
\item $x_i$: distance from node $i$ to next upstream node folllowing the bed
cross sections.
\item $Q_i$: discharge at node $i$.
\item $h_i$: water depth at node $i$.
\end{description}
\item\emph{xqz}: List of values $(x_i,Q_i,z_i)$ where:
\begin{description}
\item $x_i$: distance from node $i$ to next upstream node following the cross
sections.
\item $Q_i$: discharge at node $i$.
\item $z_i$: surface water level at node $i$.
\end{description}
\end{description}
\item\emph{Parameters}: List of required parameters to define the initial
conditions.  In the case of dry bed or steady flow, no parameter is required. In
the case of initial conditions in the form of list of values, the format of the
list is as formulated in table \ref{TabCIParametros},
\TABLE{ccc}
{
	$x_1$&$Q_1$&$h_1$\\$\vdots$&$\vdots$&$\vdots$\\
	$x_i$&$Q_i$&$h_i$\\$\vdots$&$\vdots$&$\vdots$\\
	$x_n$&$Q_n$&$h_n$
}{Format of the list of parameters required to state initial conditions as a
list of values}{TabCIParametros}
where $n$ is the number of nodes defining the initial conditions. The program
makes a linear interpolation of discharges and depths or water levels at the
nodes not matching the locations of the list and the outer points (points with longitudinal coordinate smaller than $x_1$ or greater than $x_n$) are adjusted to the values ($Q_1$, $h_1$) or ($Q_n$, $h_n$) respectively. In the case that a value
of the water surface level led to negative initial depth, the water depth would
be put to zero.
\end{description}

\subsection{Structure of the boundary condition\label{SubsecBoundary}}

In the program {\swigs} the boundary conditions must be specified according to
the structure shown in table \ref{TabFCC}.
\TABLE{cc}
{
	\multicolumn{2}{l}{$<$inlet\_flow type=``type'' properties$>$}\\
	&parameters\\
	\multicolumn{2}{l}{$<$/inlet\_flow$>$}\\
	\multicolumn{2}{l}
	{
		$<$inner\_flow type=``tipo'' name=``label'' position="position''
		propierties$>$
	}\\
	&parameters\\
	\multicolumn{2}{l}{$<$/inner\_flow$>$}\\
	\multicolumn{2}{l}{$<$outlet\_flow type=``type'' properties$>$}\\
	&parameters\\
	\multicolumn{2}{l}{$<$/outlet\_flow$>$}
}{Structure of a boundary condition}{TabFCC}

Where:
\begin{description}
\item\emph{inlet\_flow}: especifies the inflow.
\item\emph{inner\_flow}: especifies the internal boundary conditions.
\item\emph{outlet\_flow}: especifies the outflow.
\item\emph{type}: type of boundary condition.
\item\emph{name}: name of the boundary condition. Only used at internal boundary conditions. It is used as identification label and all the internal boundary conditions must have a different name. 
\item\emph{position}: is the number of the section assigned to the internal boundary condition. Only used at internal boundary conditions.
\item\emph{properties}: list of necessary properties to define the boundary condition.
\item\emph{parameters}: list of necessary parameters ti define the bounday condition.
\end{description}

The structure of the lists of parameters and properties depends on the type of boundary condition, which renders complex the definition of a general structure. The codes of types of boundary conditions admitted in the present version of the program together with their corresponding lists are: 
\begin{description}
\item\emph{q}: Fixed discharge. The list of properties is:
\begin{description}
\item\emph{discharge=``q''}: with $q$ the discharge.
\end{description}
\item\emph{h}: Fixed water depth. The list of properties is:
\begin{description}
\item\emph{depth=``h''}: with $h$ the water depth.
\end{description}
\item\emph{z}: Fixed water surface level. The list of properties is:
\begin{description}
\item\emph{height=``z''}: with $z$ the surface level.
\end{description}
\item\emph{q\_h}: Fixed discharge and water depth. The list of properties is:
\begin{description}
\item\emph{discharge=``q''}: with $q$ the discharge.
\item\emph{depth=``h''}: with $h$ the water depth.
\end{description}
\item\emph{q\_z}: Fixed discharge and water surface level. The list of properties is:
\begin{description}
\item\emph{discharge=``q''}: with $q$ the discharge.
\item\emph{height=``z''}: with $z$ the water surface.
\end{description}
\item\emph{qt}: Discharge hydrograph $Q(t)$. The list of properties consists of a data structure as in table \ref{TabCCQT}.
\TABLE{cc}
{
	$t_1$&$Q_1$\\$\vdots$&$\vdots$\\
	$t_i$&$Q_i$\\$\vdots$&$\vdots$\\
	$t_n$&$Q_n$
}{General format of the parameters list for the boundary condition of discharge hydrograph}{TabCCQT}

where:
\begin{description}
\item $t_i$: is the $i$ time in the discharge hydrograph.
\item $Q_i$: is the discharge at the $i$ time in the discharge hydrograph.
\end{description}
\item\emph{ht}: Water depth limnigraph $h(t)$. The list of properties consists of a data structure as in table \ref{TabCCHT}.
\TABLE{cc}
{
	$t_1$&$h_1$\\$\vdots$&$\vdots$\\
	$t_i$&$h_i$\\$\vdots$&$\vdots$\\
	$t_n$&$h_n$
}{General format of the parameters list for the boundary condition of water depth limnigraph}{TabCCHT}

where:
\begin{description}
\item $t_i$: is the $i$ time in the limnigraph.
\item $h_i$: is the water depth at the $i$ time in the limnigraph.
\end{description}
\item\emph{zt}: Water surface limnigraph $z(t)$. The list of properties consists of a data structure as in table \ref{TabCCZT}.
\TABLE{cc}
{
	$t_1$&$z_1$\\$\vdots$&$\vdots$\\
	$t_i$&$z_i$\\$\vdots$&$\vdots$\\
	$t_n$&$z_n$
}{General format of the parameters list for the boundary condition of water surface level limnigraph}{TabCCZT}

where:
\begin{description}
\item $t_i$: is the $i$ time in the limnigraph.
\item $z_i$: is the water level surface at the $i$ time in the limnigraph.
\end{description}
\item\emph{qt\_ht}: Discharge and water depth hydrograph $Q(t)$, $h(t)$. The list of properties consists of a data structure as in table \ref{TabCCQHT}.
\TABLE{ccc}
{
	$t_1$&$Q_1$&$h_1$\\$\vdots$&$\vdots$&$\vdots$\\
	$t_i$&$Q_i$&$h_i$\\$\vdots$&$\vdots$&$\vdots$\\
	$t_n$&$Q_n$&$h_n$
}{General format of the parameters list for the boundary condition of water depth and discharge hydrograph.}{TabCCQHT}

where:
\begin{description}
\item $t_i$: is the $i$ time in the hydrograph.
\item $Q_i$: is the discharge at the $i$ time in the hydrograph.
\item $h_i$: is the water depth at the $i$ time in the hydrograph.
\end{description}
\item\emph{qt\_zt}: Discharge and water surface level hydrograph $Q(t)$, $z(t)$. The list of properties consists of a data structure as in table \ref{TabCCQZT}.
\TABLE{ccc}
{
	$t_1$&$Q_1$&$z_1$\\$\vdots$&$\vdots$&$\vdots$\\
	$t_i$&$Q_i$&$z_i$\\$\vdots$&$\vdots$&$\vdots$\\
	$t_n$&$Q_n$&$z_n$
}{General format of the parameters list for the boundary condition of water surface level and discharge hydrograph}{TabCCQZT}

where:
\begin{description}
\item $t_i$: is the $i$ time in the hydrograph.
\item $Q_i$: is the discharge at the $i$ time in the hydrograph.
\item $z_i$: is the water surface level at the $i$ time in the hydrograph.
\end{description}
\item\emph{qh}: Rating curve $Q(h)$. The list of properties consists of a data structure as in table \ref{TabCCQH}.
\TABLE{cc}
{
	$h_1$&$Q_1$\\$\vdots$&$\vdots$\\
	$h_i$&$Q_i$\\$\vdots$&$\vdots$\\
	$h_n$&$Q_n$
}{General format of the parameters list for the boundary condition of rating curve}{TabCCQH}

where:
\begin{description}
\item $h_i$: is the water depth at the $i$ point of the rating curve.
\item $Q_i$: is the discharge at the $i$ point of the rating curve.
\end{description}
\item\emph{qz}: Gauging curve $Q(z)$. The list of properties consists of a data structure as in table \ref{TabCCQH}.
\TABLE{cc}
{
	$z_1$&$Q_1$\\$\vdots$&$\vdots$\\
	$z_i$&$Q_i$\\$\vdots$&$\vdots$\\
	$z_n$&$Q_n$
}{General format of the parameters list for the boundary condition of gauging curve}{TabCCQZ}

where:
\begin{description}
\item $z_i$: is the water surface level at the $i$ point of the gauging curve.
\item $Q_i$: is the discharge at the $i$ point of the gauging curve
\end{description}
\item\emph{supercritical}: Supercritical outflow. In this case the list of properties and parameters is empty.
\item\emph{dam}: Dam. The list of properties is:
\begin{description}
\item\emph{height="z"}: where $z$ is the overflow dam level.
\item\emph{width="w"}: where $w$ is the dam width.
\item\emph{roughness="r"}: where $r$ is the roughness coefficent at the overflow dam section.
\end{description}
and the list of associated parameters follows a structure as in table \ref{TabCCdam}.
\TABLE{cc}
{
	$t_1$&$Q_1$\\$\vdots$&$\vdots$\\
	$t_i$&$Q_i$\\$\vdots$&$\vdots$\\
	$t_n$&$Q_n$
}{General format of the data structure at a dam}{TabCCdam}

where:
\begin{description}
\item $t_i$: is the time at the $i$ point of the dam hydrograph.
\item $Q_i$: is the discharge at the $i$ point of the dam hydrograph.
\end{description}
\end{description}

The program makes a linear interpolation at any time between those given in the
list of boundary conditions.

\section{Graphical options}

{\swigs} also offers the option to choose the format of the interactive
graphical representation of the results. By pressing the button \IT{Graphical
options} from the main window or the menu option
\IT{$\underline{G}$raphical$\rightarrow\underline{O}$ptions}, a new dialog box as
the one shown in figure \ref{FigOpcionesA} appears. When the option
\IT{$\underline{M}$anual adjust} is inactive (by default), the graphical
represenation is automatically adjusted to the extrema values of the variables.
Otherwise, when it is active, the range is defined at the boxes \IT{Maximum} and
\IT{Minimum}, which can be used for a zoom view.
\fig{Options.ps}{Graphical options dialog box}{FigOpcionesA}

In this dialog box there is a block with three lables:
\begin{description}
\item[Longitudinal profile]: Represents longitudinal profiles along the
bed. The variables to represent are selected at the boxes \IT{Parameter 1} and
\IT{Parameter 2}.
\item[Time evolution]: Represents the time evolution at the cross section
selected in box \IT{Cross section} of the variables selected in the boxes
\IT{Parameter 1} and \IT{Parameter 2} (see figure \ref{FigOpcionesB}).
\item[Cross section]: Represents the water surface level at the cross
section selected in box \IT{Cross section} (see figure \ref{FigOpcionesC}).
\end{description}
\fig{Options2.ps}{Time evolution graphical options}{FigOpcionesB}
\fig{Options3.ps}{Cross section graphical options}{FigOpcionesC}

Box \IT{Animation} allows the selection of the dynamic way to represent the
data. The options are:
\begin{description}
\item[No animation]: Only the results at the final time at represented. This is
the quickest option but the dynamical evolution cannot be seen.
\item[Quick animation]: The results obtained every \IT{measuring interval} are
represented. It is an optimal option since the dynamic evolution is shown not
affecting the speed of calculation.
\item[Detailed animation]: The solution is represented every time step of
computation. This can lead to slow simulations in some cases.
\end{description}

It must be signaled that all the options as well as the represented variables
and their ranges can be altered during the simulation.

\section{The numerical simulation}

Once the bed form, inital conditions and boundary conditions are defined, the
basic parameters for the simulation are:
\begin{description}
\item[Maximum cell size]: Maximum size of the mesh cell.
\item[Simulation time]: Desired period of simulation.
\item[Measuring interval]: Fixed time interval used to save intermediate
results.
\item[CFL number]: Courant-Friedrichs-Lewy number. It controls the size of the
time step once the size of the cells is fixed.
\end{description}

A number of cells larger than the number of bed cross sections can be used. In
this case, the program automatically generates the required intermediate
sections. This can be necessary in long rivers defined by a limited number of
sections and also in specially difficult situations such as low discharges over
rough surfaces requiring a mesh refinement.

The $CFL$ number is used to control the size of the time step $\Delta t$ from
the law $\Delta t=CFL\min\left(\frac{\Delta x}{|u|+\sqrt{gA/B}}\right)$ where
$u$ is the flow velocity, $g$ is the gravity acceleration, $A$ is the wetted
cross section and $B$ is the top water width. The value $CFL=0.9$ is used by
default, since it optimizes both the calculation time and accuracy. For values
$CFL\leq 0.9$, the program automatically uses the second order in space and time
TVD upwind scheme. For larger values, it automatically applies an
unconditionally stable first order semi-explict upwind scheme.

The value of this numerical parameter is very relevant to the quality of the
numerical solution. The following values are recommended:
\begin{description}
\item[Maximum accuracy and stability (slow calculation)]: $CFL=0.9$.
It is worth recalling that the computational speed is directly proportional to
the $CFL$ value and therefore this option can slow down the simulation. In
highly rough beds such as mountain creeks with low discharges it may turn
necessary to increase the number of computational cells and reduce that of the
$CFL$ number to control the stability of the solution.
\item[Stability and meduim accuracy (quick)]: The $CFL$ number can be increased
to a value about two orders of magnitud (100 times) smaller than the number of
cells keeping a resonable accuracy and stabilioty except in very irregular and
dry situations.
\item[Maximum speed]: The maximum computational speed can be achieved by
increasing the $CFL$ to a value about 10 times less than the number of
computational cells. Some accuracy in the propagation of transients is lost in
this case, but smooth steady solutions are obtained with high accuracy. It isnot
recommended either for steady solutions in very irregular beds including
transcritical situations.
\end{description}

As an extra recommendation in case of dealing with very steep, rough and
irregular beds, the maximum space interval size should be controlled by
\[\Delta x\leq\frac{h^\frac{4}{3}}{gn^2}\]
with $h$ the water depth, $g$ the acceleration of gravity and $n$ the Manning
roughness coefficient.

Once the bed geometry, numerical scheme and graphical options are selected, the
button \IT{Start simulation}  or the menu option
\IT{$\underline{S}$imulate$\rightarrow\underline{S}$tart} activates the
calculation.

The results are interactively plotted in a main graphic window with the format
specified. The form of the plot can be seen in figure \ref{FigSimulacion}.
\fig{Simulation.ps}{Simulation plot}{FigSimulacion}

Apart form this plot, the lower toolbar also displays 3 boxes showing:
\begin{description}
\item[Simulation time]: Simulation time.
\item[Computational time]: Time used for the computation.
\item[Conservation error]: Percentage of mass conservation error generated.
\end{description}

At any time after the simulation has started, the calculation can be stopped by clicking on \IT{Stop simulation} or the menu
\IT{$\underline{S}$imulate$\rightarrow$Sto$\underline{p}$}.

\section{Results file}

Every \IT{measuring interval} the program saves the partial results in a binary
format file called \IT{sol.tmp}. From this file, two kind of results can be
obtained in several column ASCII format: longitudinal profile files and time evolution files. 

Once the simulation has finished, by clicking on \IT{Save solution} or the menu \IT{$\underline{S}$imulate$\rightarrow$Sa$\underline{v}$e solution}, the dialog box shown on figure \ref{FigGuardar} is created and allows to save the solution ina file.
\fig{Save.ps}{Dialog box to save the solution}{FigGuardar}
This box offers the choice between \IT{Longitudinal profile}, saving the longitudinal profile at a given time and \IT{Time evolution}, saving a time evolution file for a given cross section. The format of these files is described next. 

\subsection{Longitudinal profiles}

The longitudinal profile files is written in 11 column format as shown on
table \ref{TabPerfil}
\TABLE{cccccc}
{
	$x_1$&$p_{1,1}$&$\dots$&$p_{1,j}$&$\dots$&$p_{1,10}$\\
	$\vdots$&$\vdots$&&$\vdots$&&$\vdots$\\
	$x_i$&$p_{i,1}$&$\dots$&$p_{i,j}$&$\dots$&$p_{i,10}$\\
	$\vdots$&$\vdots$&&$\vdots$&&$\vdots$\\
	$x_n$&$p_{n,1}$&$\dots$&$p_{n,j}$&$\dots$&$p_{n,10}$
}{Format of the longitudinal profile files}{TabPerfil}
where $x_i$ is the longitudianl coordinate of node $i$ and the parameters
$p_{i,j}$ represent the following variables at that node:
\begin{description}
\item $\mathbf{p_{i,1}}$: Discharge.
\item $\mathbf{p_{i,2}}$: Water depth.
\item $\mathbf{p_{i,3}}$: Wetted cross section area.
\item $\mathbf{p_{i,4}}$: Cross section top water width.
\item $\mathbf{p_{i,5}}$: Water surface level.
\item $\mathbf{p_{i,6}}$: Cross section bottom level.
\item $\mathbf{p_{i,7}}$: Banks level.
\item $\mathbf{p_{i,8}}$: Flow velocity.
\item $\mathbf{p_{i,9}}$: Critical velocity.
\item $\mathbf{p_{i,10}}$: Froude number.
\end{description}

\subsection{Time evolution}

At a given cross section, the time evolution is written in a 11 column format as
specified in table \ref{TabEvolucion}
\TABLE{cccccc}
{
	$t_1$&$p_{1,1}$&$\dots$&$p_{1,j}$&$\dots$&$p_{1,10}$\\
	$\vdots$&$\vdots$&&$\vdots$&&$\vdots$\\
	$t_i$&$p_{i,1}$&$\dots$&$p_{i,j}$&$\dots$&$p_{i,10}$\\
	$\vdots$&$\vdots$&&$\vdots$&&$\vdots$\\
	$t_n$&$p_{n,1}$&$\dots$&$p_{n,j}$&$\dots$&$p_{n,10}$
}{Format of the time evolution files}{TabEvolucion}
where $t_i$ is the $i$ time and the parameters $p_{i,j}$ represent the following
variables at that time:
\begin{description}
\item $\mathbf{p_{i,1}}$: Discharge.
\item $\mathbf{p_{i,2}}$: Water depth.
\item $\mathbf{p_{i,3}}$: Wetted cross section area.
\item $\mathbf{p_{i,4}}$: Cross section top water width.
\item $\mathbf{p_{i,5}}$: Water surface level.
\item $\mathbf{p_{i,6}}$: Cross section bottom level.
\item $\mathbf{p_{i,7}}$: Banks level.
\item $\mathbf{p_{i,8}}$: Flow velocity.
\item $\mathbf{p_{i,9}}$: Critical velocity.
\item $\mathbf{p_{i,10}}$: Froude number.
\end{description}

\end{document}
